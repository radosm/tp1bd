
\documentclass[a4paper,10pt]{article}
\addtolength{\textwidth}{5.2cm}
\addtolength{\voffset}{-3cm}
\addtolength{\hoffset}{-2.5cm}
\addtolength{\textheight}{5cm}
\addtolength{\headheight}{14.0pt}
\usepackage[utf8]{inputenc}
\usepackage[spanish,activeacute]{babel} 
\usepackage{paralist}
\usepackage[pdftex]{graphicx}
\usepackage{epsfig}
% \usepackage[T1]{fontenc}
% \usepackage{cmap}
% \usepackage{fix-cm}
% \usepackage{lscape}
% \usepackage{amsmath}
% \usepackage{amssymb}
% \usepackage{mathrsfs}
% \usepackage{epstopdf}
% % \usepackage{algorithmic}
\usepackage{verbatim}
% \usepackage[lined,boxed,commentsnumbered]{algorithm2e}
% \usepackage{graphicx}
% \usepackage[utf8]{inputenc}

% \usepackage{caratulayabs} %Para la caratula
\usepackage[pdfborder={0,0,0}]{hyperref}
% \usepackage{ifthen}
% \usepackage{paralist}
% \usepackage{ulem}

\newcommand{\dotu}{\bgroup \markoverwith{\lower .4ex\hbox{\_}}\ULon} % para subrallado punteados

\newcommand{\header}[1]{\textsf{#1}}
\newcommand{\todo}[1]{\frame{\textsf{TODO} #1}}

%Redefino algunos nombres (como uso babel, lo tengo que hacer asi):
% Donde dice \chapter quiero que aparezca Problema en lugar de Capitulo
\addto\captionsspanish{% esto lo necesito porque uso babel
  \renewcommand{\chaptername}%
    {Cap\'itulo}%
}

%Donde dice Indice general, quiero que aparezca Contenidos
\addto\captionsspanish{% esto lo necesito porque uso babel
  \renewcommand{\contentsname}%
    {Contenidos}%
}

%Donde dice Indice general, quiero que aparezca Contenidos
\addto\captionsspanish{% esto lo necesito porque uso babel
  \renewcommand{\appendixname}%
    {Anexo}%
}

%Donde dice Cuadro, quiero que aparezca Tabla
\addto\captionsspanish{% esto lo necesito porque uso babel
  \renewcommand{\tablename}%
    {Tabla}%
}

\DeclareMathAlphabet{\mathpzc}{OT1}{pzc}{m}{it}

\input ulem.sty

\begin{document}



% 	% begin caratula
% 		\materia{Teoría de las Comunicaciones}
% 		\submateria{Segundo Cuatrimestre de 2012}
% 		\titulo{Taller de Capa de Transporte}
% 		\integrante{Martin Epszteyn}{198/08}{poplpops@hotmail.com}
% 		\integrante{Alejandro Dan\'os}{381/10}{adp007@msn.com}
% 		\integrante{Mauricio Alfonso}{65/09}{mauricioalfonso88@gmail.com}
% 		\maketitle
% 	% fin caratula
	
% 	\tableofcontents

\newpage
\newpage


\section{Modelo Relacional}

\begin{itemize}
  \item \texttt{Cuenta(profesión, email, clave, \underline{userid}, apellido, nombre, teléfono, dirección)
  \\PK = CK = \{userid\}
  \\FK = \{\} 
  }

  \item \texttt{Tarjeta(\dotu{userid}, \underline{numeroTarjeta}, marca, banco\_emisor, validez, dirección\_facturación)
  \\PK = CK = \{numeroTarjeta\}
  \\FK = \{userid\}
  }

  \item \texttt{Reserva(\dotu{userid, codigo\_clase}, \underline{codigo\_reserva}, dirección\_entrega, fecha, forma\_de\_pago, estado)
  \\PK = CK = \{codigo\_reserva\}
  \\FK = \{userid, codigo\_clase\} 
  }

  \item \texttt{persona\_reserva(\underline{\dotu{tipo\_doc, nro\_doc, codigo\_reserva}})
  \\PK = CK = FK = \{(tipo\_doc, nro\_doc, codigo\_reserva)\}
  }

  \item \texttt{Persona(\underline{tipo\_doc, nro\_doc}, fecha\_nac, nacionalidad, nombre, apellido)
  \\PK = CK = \{(tipo\_doc, nro\_doc)\}
  \\FK = \{\}
  }

  \item \texttt{reserva\_viaje(\underline{orden, \dotu{codigo\_reserva, numero\_vuelo, fecha\_viaje}})
  \\PK = CK = \{(orden, codigo\_reserva, numero\_vuelo, fecha\_viaje)\}
  \\FK = \{codigo\_reserva, numero\_vuelo, fecha\_viaje\}}

  \item \texttt{Viaje(\underline{fecha\_viaje, \dotu{numero\_vuelo}})
  \\PK = CK = \{(fecha\_viaje, numero\_vuelo)\}
  \\FK = \{numero\_vuelo\}}

  \item \texttt{Vuelo(\underline{numero\_vuelo}, hora\_despegue, duración, millas, \dotu{origen, destino})
  \\PK = CK = \{numero\_vuelo\}
  \\FK = \{origen, destino\}}

  \item \texttt{cronograma(\underline{\dotu{día, numero\_vuelo}}, \dotu{codigo\_configuracion})
  \\PK = CK = \{(día, número\_vuelo)\}
  \\FK = \{día, numero\_vuelo, codigo\_configuracion\}}
  \\-----LA PK ESTA DIFERENTE QUE EN LA BASE!!!

  \item \texttt{Dia(\underline{día})
  \\PK = CK = \{día\}
  \\FK = \{\}}

  \item \texttt{Configuración(\underline{codigo\_configuración}, descripción)
  \\PK = CK = \{codigo\_configuración\}
  \\FK = \{\}}
  \\-----CORREGIR CONFIGURACION - ASIENTOS - CLASE!!

  \item \texttt{asientos(\underline{\dotu{codigo\_configuracion, codigo\_clase}}, cantidad)
  \\PK = CK = FK = \{(codigo\_configuracion, codigo\_clase)\}}

  \item \texttt{Clase(\underline{codigo\_clase}, descripción)
  \\CK = \{codigo\_clase, descripcion\}
  \\PK = \{codigo\_clase\}
  \\FK = \{\}}

  \item \texttt{Precio(\underline{\dotu{numero\_vuelo, codigo\_clase}}, tarifa)
  \\PK = CK = FK = \{numero\_vuelo, codigo\_clase\}}

  \item \texttt{config\_modelo(\underline{\dotu{codigo\_configuracion, codigo\_modelo}})
  \\PK = CK = FK = \{codigo\_configuracion, codigo\_modelo\}}

  \item \texttt{Modelo\_avion(\underline{codigo\_modelo}, descripcion)
  \\PK = CK = \{codigo\_modelo\}
  \\FK = \{\}}

  \item \texttt{Avion(\underline{codigo\_avion}, año\_fabric, \dotu{codigo\_modelo, codigo\_pais})
  \\PK = CK = \{codigo\_avion\}
  \\FK = \{codigo\_modelo, codigo\_pais\}}
  
  \item \texttt{tripulacion(\underline{\dotu{fecha\_viaje, numero\_vuelo, codigo\_tripulante}}, \dotu{codigo\_avion})
  \\PK = CK = \{\}
  \\FK = \{\}}
  \\-----FALTA VER COMO VAN LAS CLAVES!!

  \item \texttt{Tripulante(\underline{codigo\_tripulante}, nombre, apellido, cargo)
  \\PK = CK = \{codigo\_tripulante\}
  \\FK = \{\}}

  \item \texttt{Pais(\underline{codigo\_pais}, nombre)
  \\CK = \{codigo\_pais, nombre\}
  \\PK = \{codigo\_pais\}
  \\FK = \{\}}

  \item \texttt{Aeropuerto(\underline{codigo\_aeropuerto}, información\_transporte, tasa, nombre, \dotu{codigo\_pais})
  \\CK = \{codigo\_aerop, nombre\}
  \\PK = \{codigo\_aerop\}
  \\FK = \{codigo\_pais\}}

  \item \texttt{Telefonos\_aeropuerto(\underline{numero, \dotu{codigo\_aeropuerto}})
  \\PK = CK = \{(numero, codigo\_aeropuerto\}
  \\FK = \{codigo\_aeropuerto\}}

\end{itemize}

\end{document}